% Chapter 1
\chapter{Tests performed} % Main chapter title

\label{sec:tests} % For referencing the chapter elsewhere, use \ref{Chapter1} 

\section{Test model: lung cancer}

\begin{itemize}
	\item Total of 9 matrices (one per age group: 35-39, 40-44, …, 75-79).
	\item 7 health states: healthy, stages I-II, stage IIIa, stage IIIb, LC survival, death from LC, death from other causes → 7x7 matrices. Sometimes the LC survival state is excluded from the calibration resulting in 6x6 matrices.
	\item Matrices represent monthly steps in the simulation. Since they are applied for 5-year groups, each matrix is used in 5 * 12 = 60 iterations in the model.
	\item A simplified calibration can be performed without running the model, only the matrices are used. This is a fast approximation since we are not considering some factors of the full model (e.g. prevalence of smoking):
	\begin{itemize}
		\item LC survival state is excluded → 6x6 matrices
		\item From the 6x6=36 probabilities per matrix, only 11 probabilities are allowed to change. The rest are either constant (zeroes, ones) or one minus the sum of the rest of the row.
		\item The error measurement is a weighted sum of the absolute differences of the LC incidence, LC mortality and mortality from other causes. The weights are 0.45, 0.45 and 0.10 respectively.
	\end{itemize}
\end{itemize}

\subsection{Simplified calibration, 1 matrix}

\begin{itemize}
	\item Source file: models/lung/calibration\_wrapper.R (N\_MATRICES set to 1)
	\item Only the first age group is being calibrated (35-39): 1x11 = 11 parameters.
\end{itemize}

\begin{table}[h]
	\begin{tabular}{p{2cm}|l|l|l|l}
		\textbf{Algorithm} 		& \textbf{Initial matrix} & \textbf{Nelder-Mead} 		& \textbf{Particle swarm} 	& \textbf{Bayesian} \\
		\hline \\
		\textbf{Error}	& 1.1545799674960& 0.6633085653748	& \cellcolor{green}0.66298515		& \cellcolor{green}0.6629851533965 \\
		\textbf{Time (s)} & - & \cellcolor{green}0.76 & 22.97 & \cellcolor{red!20}114.89 \\
		\textbf{Evaluations} & - & 252 & \cellcolor{red!20}10100 & \cellcolor{green}21 \\
	\end{tabular}
\end{table}

\subsection{Simplified calibration, 2 matrices}

\begin{itemize}
	\item Source file: models/lung/calibration\_wrapper.R (N\_MATRICES set to 2)
	\item The first and second age groups are being calibrated (35-39 and 40-44): 2x11 = 22 parameters.
\end{itemize}


\begin{table}[h]
	\begin{tabular}{p{2cm}|l|l|l|l}
		\textbf{Algorithm} 		& \textbf{Initial matrix} & \textbf{Nelder-Mead} 		& \textbf{Particle swarm} 	& \textbf{Bayesian} \\
		\hline \\
		\textbf{Error}	& 1.6495138536869 & 0.7333381543348456 & \cellcolor{green}0.72801101 & \cellcolor{green}0.7287116136105645 \\
		\textbf{Time (s)} & - & \cellcolor{green}4.44 & 24.46 & \cellcolor{red!20}421.69 \\
		\textbf{Evaluations} & - & 2092 & \cellcolor{green}10100 & \cellcolor{green}70 \\
	\end{tabular}
\end{table}

\subsection{Simplified calibration, all (9) matrices}
\begin{itemize}
    \item Source file: models/lung/calibration\_wrapper.R (N\_MATRICES set to 9)
	\item All age groups are being calibrated: 9x11 = 99 parameters.
	\item Standard bayesian optimization takes too much time and the process was aborted before completion. Other strategies could be attempted: calibrate matrices sequentially, restrict number of parameters, optimize gaussian process regression (see section \ref{sec:proposal}), ...
\end{itemize}

\begin{table}[h]
	\begin{tabular}{p{2cm}|l|l|l|l}
		\textbf{Algorithm} 		& \textbf{Initial matrix} & \textbf{Nelder-Mead} 		& \textbf{Particle swarm} 	& \textbf{Bayesian} \\
		\hline \\
		\textbf{Error}	& 6.6842251473203 & 4.0210661197016	& 4.3594607 & \cellcolor{red}<Aborted> \\
		\textbf{Time (s)} & - & 57.58 & 35.64 & \cellcolor{red}- \\
		\textbf{Evaluations} & - & 19800 & 9407 & \cellcolor{red}- \\
	\end{tabular}
\end{table}